%label:"exm:linearSymplecticSpace"
%type:"example"
%name:"linear symplectic space"


    The simplest example comes from $\RR^{2n}$, which we give the coordinates $(q_i, p_i)$. 
    In these local coordinates, we can define a symplectic form by  
        \[\omega_{std}=\sum_{i=1}^n d p_i\wedge d q_i.\] 
    Note that when $n=1$, this gives the standard area form on $\RR^2$.
    In these coordinates, it is easy to check that $\frac{\omega_{std}^n}{n!}=\text{vol}_{\RR^{2n}}$, the standard volume form.

