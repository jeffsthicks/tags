%label:"art:tropicalGeometryIntroduction"
%type:"article"
%name:"a short introduction to tropical geometry"


We first study the log map 
\[
\Log_t:= \log_t|-|: \CC\to \RR\cup\{-\infty\}
\]
where $t>1$. 
and try to define a ``ring structure'' on $\RR\cup\{-\infty\}$ which makes this log map a homomorphism. 
The first guess that one would take is to define 
\begin{align*}
    ``\times" \text{ given by the operation  } q_1 ``\times" q_2  = &\Log_t(\Log_t^{-1}(q_1) \cdot \Log_t^{-1}(q_2)) = x+y \\
    ``+"    \text{ given by the operation } q_1 ``+" q_2  =    &\Log_t(\Log_t^{-1}(q_1) + \Log_t^{-1}(q_2))
\end{align*}
While the first operation is well defined, the second is not! In order to make this well defined we take the limit of the second equation as $t\to \infty$, from which we obtain 
\[\lim_{t\to\infty} \Log_t(\Log_t^{-1}(q_1) + \Log_t^{-1}(y)) = \max(q_1, y).\]
This gives us a definition for our new operations. 
%label:"def:tropicalSemiField"
%type:"definition"
%name:"tropical semi-field"


The \emph{semi-field of tropical numbers} is the set $\TT:= \RR\cup_\infty$ equipped with the operations (called tropical plus and tropical times):
\begin{align*}
    q_1\oplus q_2 =& \max(q_1, q_2)\\
    q_1\odot  q_2 =& q_1+q_2
\end{align*}
where $q_1, q_2\in \TT$.
The goal is now to understand how algebraic geometry over $\CC$ relates to ``algebraic geometry'' over the semi-field $\TT$. The first thing to do it to exchange polynomials for tropical polynomials. For instance, given $f(z_1, z_2): \CC^2\to \CC$ a polynomial of two variables, we declare the tropicalization of the polynomial to be 
\[f(z_1, z_2)=\sum a_{ij} z_1^iz_2^j \leftrightarrow \TropB(f)(q_1, q_2):= \max (a_{ij}+iq_1 + jq_2 )\]
%label:"exm:tropicalization"
%type:"example"
%name:"tropicalization"


If $f=q_1+q_2+1$, the tropicalization is given by $\TropB(f) = \max(1+q_1, 1+q_2, 1)$


