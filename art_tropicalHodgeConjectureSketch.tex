%label:"art:tropicalHodgeSketch"
%type:"article"
%name:"the tropical Hodge conjecture"
%parent:"art_tropicalGeometrySeminarOutline"



Associated to $X$ a tropical variety, we want to some geometry. We have a short exact sequence 
\[ 0 \to \RR_X\to PA_X\to \Omega_X\to 0\]
where $\Omega_X$ gives the tropical cotangent sheaf, whose sections are called tropical 1-forms. Another characterization is: given a fan $\Sigma\subset \RR^n$, want to define $F_k(\Sigma)\subset \bigwedge^k\ZZ^n$ the set of polyvectors generated by $v_1\wedge \cdots \wedge v_k$, where $v_1, \ldots, v_k$ belong to a single cone of $\Sigma$. This associates to $X$ a tropical variety $F_k$ a constructible sheaf on $X$ so that $F_{k, x}=F_k(\Sigma(x))$, where $\Sigma(x)$ is the tangent cone at $x$. This defines for us a chain complex 
\[C_\bullet(X, F_k)\]
which comes with a tropical $(p, q)$ homology which is the homology $H_q(X, F_p)$.  When $X$ can be geometrically realized, this compute some limit of a mixed Hodge structure, and satisfies the K\"ahler/Hodge package. Finally, there exists a cycle class map 
\begin{align*}
    CH_k(\RR^n)\to H_{p}(\RR^n, F_p)&&
    Z\mapsto \Sigma_\subset K_{cell} \weight(\Delta)\cdot v_\Delta \otimes \Delta
\end{align*}
where $\Delta$ is the volume of the $k$-cell. The tropical Hodge conjecture is that this is an isomorphism. 
