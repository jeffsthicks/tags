%label:"exm:compactnessFromMaximumModulus"
%name:"maximum modulus provides compactness"
%type:"example"



    The maximum modulus principle states that if $\phi: D^2\to \CC$ is a holomorphic function from the disk to $\CC$, that the maximum of $|\phi|: D^2\to \RR_{\geq 0}$ is achieved on $\partial D^2$.

    Let $\hat X$ be a non-compact symplectic manifold with compatible almost complex structure $J$, along with a $J-\jmath$-holomorphic projection $W: \hat X\to \CC$. Suppose that the fibers of $W$ are compact. Pick two loops $\gamma_-, \gamma_+\subset \hat X$ and $r_0\in \RR$ large enough so that $U:=W^{-1}(\{z \st |z|\leq r\})$ contains $\gamma_-, \gamma_+$. We will prove that every pseudoholomorphic cylinder $u: S^1\times \RR\to \hat X$ with ends limiting to $\gamma_-, \gamma_+$ has image contained within the compact subset $U$. 

    The composition $W\circ u: S^1\times \RR\to \CC$ is a holomorphic map, with ends limiting to $W(\gamma_\pm)$, and therefore satisfies the maximum modulus principle. Since the boundary is sent to $W(\gamma_\pm)$, we obtain that $|W|$ achieves a value no greater than $r_0$ on $u$; therefore $\Im(u)\subset U$. It follows that the image of $u$ is contained within a compact set. 
    \label{exm:compactnessFromMaximumModulus}
