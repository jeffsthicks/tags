%label:"sol:symplecticCohomologyCotangentBundleOfS1"
%author:AlvaroMuniz
%name:"symplectic cohomology of $T^*S^1$"
%type:"solution"

 
\begin{enumerate}
    \item A straighfoward computation shows that  $\varphi_t(q,p)=(q-2pt,p)$. Therefore the flow is 1-periodic if and only if
    $2p=2k\pi, k\in \ZZ$, i.e. $p=k\pi$.
    \item At the zero-section $S^n\subset T^*S^n$ we have $X_H\restr{S^n}=0$. Consider therefore a {\it perturbed} Hamiltonian $\tilde H=H+\epsilon h$, where $h(q,p):=\eta(p)\sin q$ with $\eta:\RR\to \RR$  a cut-off function centered at $p=0$. Note that $\tilde H=H$ whenever $\eta=0$, so outside a neighborhood of the zero-section we have a bijection between orbits of $\tilde H$ and orbits of $H$, all of which are non-constant. Furthermore, we have $X_{\tilde H}=(-2p+\epsilon\eta(p)\cos q)\partial_q +\epsilon\eta'(p)\sin q\partial_p$ from which we deduce that, for $\epsilon$ small enough, the only constant orbits occur when $p=0$ and $\cos q=0$, a total of two points. It remains to argue that there cannot be other non-constant orbits in this region. This follows from the fact that we can make $|X_{\tilde H}|$ as small as we want --- again, by choosing suitably small $\epsilon$ and cut-off width --- so that any periodic orbit must necessarily have period $T>>1$.
    
    \item We will use the time-dependent Hamiltonian $\tilde H_t$ to compute the symplectic cohomology $SH^\bullet(T^*S^1)$. Note $\tilde H_t$  has:
    \begin{itemize}
        \item Two constant orbits ocurrying at $(q,p)=\left(\pm \frac{\pi}{2},0\right)$, call them $\gamma^0_{max,min}$;
        \item Two non-constant orbits $\gamma_{max,min}^k$ for each $k\in \ZZ\setminus\set{0}$ ocurrying near $p_k=k\pi$.
    \end{itemize}
    These are by definition the generators of $FC^\bullet(\tilde H_t)$. Now, for topological reasons (homology/homotopy class of the orbits) we can exclude any pseudoholomorphic cylinders between $\gamma^k_*$ and $\gamma^l_*$ for $k\neq l$. Furthermore, by assumption there is also no differential between $\gamma^k_+$ and $\gamma^k_-$ for any $k$. Putting both things together we conclude the differential vanishes, and thus each orbit defines a generator of $SH^\bullet(T^*S^1)$. In conclusion, we have
    \[    SH^\bullet(T^*S^1)=\bigoplus_{\substack{k\in\ZZ \\ *=max,min}}\RR\cdot \gamma^k_*\cong \RR[x]\oplus\RR[x]\]
    
    \item We first note that  $\pi_0(LS^1)=\ZZ$, where $k\in \ZZ$ distinguishes homotopy classes. Let $LS^1(k)\subset LS^1$ be the connected component consisting of loops of homotopy class equal to $k\in\pi_1(S^1)\cong\ZZ$. Then we have a splitting
    \[    H_*(LS^1)\cong\oplus_{k\in \ZZ}H_*(LS^1(k))\]
    There is an obvious inclusion $S^1\into LS^1(k), p\mapsto (t\mapsto pe^{i2\pi k t})$, and one can see the map $LS^1(k)\to S^1$ that records the base-point of the loop is a homotopy inverse. Therefore $S^1\simeq LS^1(k)$ have the same homology, and we conclude
    \[    H_*(LS^1)=\oplus_{k\in \ZZ} \RR^2\cong \RR[x]\oplus\RR[x]\]
    \item Each component $LS^1(k)$ contributes two generators to $H_*(LS^1)$. Consider a generator of $H_*(LS^1)$ coming from  $H_0(LS^1(k))$, which is just a point $\gamma\in LS^1(k)$. Then there is an obvious 1-cycle $\Gamma\in C_1(LS^1(k)), \partial\Gamma=0$ that we can build, namely  $\Gamma(s)=\gamma(\cdot+s)$, the 'rotation' of the loop $\gamma$. Note that recording the basepoint $\Gamma(s)(0)=\gamma(s)$ of each rotated loop we obtain the fundamental cycle of $S^1$, so according to our identification $S^1\simeq LS^1(k)$ the 1-cycle $\Gamma$ is precisely the degree one generator of $H_*(LS^1(k))$.
\end{enumerate}