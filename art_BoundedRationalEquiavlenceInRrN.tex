%label:"art:BoundedRationalEquiavlenceInRrN"
%type:"article"
%name:"bounded rational equiavlence in $\RR^n$"
%caption:""
%parent:"art_TropicalChowGroupsII"


%label:"def:TropicallyNumericalEquivalenct"
%type:"definition"
%name:"tropically numerical equivalenct"
%caption:""
%parent:"art_BoundedRationalEquiavlenceInRrN"


    Let $Q\subset \RR^n$ be a cycle of codimension $k$. We obtain a map 
    \begin{align*}
        d_Q: Z_k(\RR^n)\to \ZZ\\
        P\mapsto \deg(P\cdot Y)
    \end{align*}


We say that $Q_1$ is numerically equivalent to $Q_2$ if $d_{Q_1}=d_{Q_2}$.
\end{definition}
Observe that if $Q_1\sim^b Q_2$, this is the same as saying that $Q_1-Q_2\sim^b 0$, which implies that $d_{Q_1}-d_{Q_2}=0$. It is natural to ask if the converse hold. We first understand the question locally, i.e. the intersection of fans. 
%label:"prp:DDeterminesAFan"
%type:"proposition"
%name:"d determines a fan"
%caption:""
%parent:"art_BoundedRationalEquiavlenceInRrN"


    If $Q_1, Q_2\subset \RR^n$ are fans in $\RR^n$, then $d_{Q_1}=d_{Q_2}$ tells us that $Q_1 = Q_2$. 


\begin{corollary}
    If $Q_1\sim^b Q_2$ are bounded equivalent fans, they are the same fan. 
\end{corollary}
The plan: 
\begin{enumerate}
    \item Associate to a cycle $Q$ a fan $Q'$,
    \item Show that $Q\sim^b Q'$, 
    \item use the previous results. 
\end{enumerate}

For the first part: Let $\sigma\subset \RR^n$ be the polyhedron. Define the recession cone 
\[\text{Rec}(\sigma):= \{v\in \RR^ \st q+\RR_{\geq 0} v \subset \sigma \forall q\in \sigma\}\]
For sufficiently fine polyhedral structure on $Q$, one can show that the $\text{Rec}(X):=\{\text{Rec}(\sigma), \sigma\in X\}$ comes with the structure of a balanced fan.  This essentially is just ``zooming out'' really far away from your variety. 
%label:"thm:CyclesAndRecessionCones"
%type:"theorem"
%name:"cycles and recession cones"
%caption:""
%parent:"art_BoundedRationalEquiavlenceInRrN"


    Let $Q\subset \RR^n$ be a cycle. It is bounded rationally equivallent to its recession cone, that it: $Q\sim^b \text{Rec}(Q)$


The main proof of the theorem is the following. 
%label:"prp:TropicalCyclesAsSums"
%type:"proposition"
%name:"tropical cycles as sums"
%caption:""
%parent:"art_BoundedRationalEquiavlenceInRrN"


    Any cycle $Q$ can be written as a sum, 
    \[Q=\sum(Q_i+\vec v_i)\]
    where each $Q_i$ is a fan. 


%label:"prf:TropicalCyclesAsSums"
%type:"proof"
%name:"tropical cycles as sums"
%caption:""
%parent:""


    Every $q\in Q$ has a neighborhood $\text{Star}_Q(p)$ which looks like a fan. For any fan $P$, we define its splitting dimension $\dim_s(P)$ to be the maximum $k$ so that 
    \[P=\sum P_i\]
    where the $P_i$ have at least $k$-dimensional translation invariance. We then define $\dim_s(q):= \dim_s(\text{Star}_Q(p)$. 
    The proof then proceeds by removing first the star fans of points with star fan dimension 0, and then proceeding through the tropical subvariety. 


We obtain the proof of the theorem:
\begin{align*}
    \text{Rec}(Q)=& \text{Rec}(\sum Q_i + \vec v_i) \\
    = &\sum \text{Rec}(Q_i + \vec v_i) = \sum Q_i \sim^b \sum Q_i+\vec v_i = Q
\end{align*}
In summary, the following are equivalent:
\begin{align*}
    Q_1\sim^b Q_2 && d_{Q_1} = d_{Q_2} && \text{Rec}(Q_1)= \text{Rec}(Q_2)

