%tag:000Y
%label:art:triangulatedCategories
%author:JeffHicks
%name:"triangulated categories"
%type:article


Broadly speaking, many of the structural results which make algebra such a useful tool come from the identifications between maps between objects, and the objects themselves. For example, in the setting of abelian groups, we can associate to every morphism $f: A\to B$ a kernel, image, and cokernel. Because this language is so useful, there is a whole language of \emph{abelian categories} whose morphisms come with the data of kernels, images, and cokernels.

In many categories there is not a reasonable way to construct the ``kernel'' of a morphism. For example, given a continuous map $f: X\to Y$ between two topological spaces, we have no reasonable definition of a kernel of a continuous map. There is, however, the notion of a cone $\operatorname{cone}(f)$, which remembers which points in $Y$ came from the image of $f$. 
%tag:0000
%label:"def:topologicalMappingCone"
%author:JeffHicks
%name:"topological mapping cone"
%type:"definition"

 
Given \(f: X\to Y\) a continuous map, the \emph{cone} of \(f\) is the space 
\[\text{cone}(f):= (X\times I)\cup Y / \sim,\]
where the equivalence identifies the points $(x_1, 0)\sim (x_2, 0)$ and $(x, 1)\sim f(x)$.
We have continuous maps  \(X\xrightarrow{f} Y \xrightarrow{i} \text{cone}(f)\).
 
%tag:0001
%label:rem:topologicalMappingCone
%author:JeffHicks
%name:"some properties of topological mapping cone"
%type:remark
%parent:0000

 
    In the homotopy category of pointed spaces, the mapping cone takes a special meaning. Let $f: (X,x)\to (Y, y)$ be a pointed map. From this we can form $(Z, z):=(\cone(f), y)$ which is again a pointed space. We now address some relations between spaces $(X, x), (Y, y)$ and $(Z, z)$.
\begin{itemize}
    \item Observe that the composition \(i\circ f: (X, x)\to (Z, x))\) is homotopic to the constant map. In the homotopy category of pointed spaces, we can therefore write $i\circ f \sim 0$, where $0: (X, x)\to (Z, x)$ is the map factoring through a point.
    \item We can additionally look at the cone  
    \[(Y, y)\xrightarrow{i}(Z, z).\]
    This second cone can be rewritten in terms of the data $f, X$ and $Y$ as
    \[ (Y\times J)\cup ((X\times I)\cup Y / \sim\]
    where the relations are 
    \[(x_1, 0)\sim (x_2, 0) , (x, 1)\sim f(x)) , (y_1, 0)\sim (y_2, 0), (y, 1)\sim y.\]
    This is homotopic to the \snip{suspension}{def:suspension} $\Sigma X$ allowing us to write the ``long exact sequence''
    \[(X, x)\to (Y, y)\to (Z, z)\to (\Sigma X, x)\to (\Sigma Y, y)\to \cdots\]
    in the homotopy category of pointed spaces.
\end{itemize}

 
A triangulated category is a set of axioms for a category which encapsulates many of the properties of the cone construction. 
%tag:0008
%label:"def:triangulatedCategory"
%author:JeffHicks
%name:"triangulated Category"
%type:"definition"

 
 
A triangulated category is an \emph{additive} category $\mathcal C$, along with the structure of 
\begin{itemize}
    \item an additive automorphism $\Sigma: \mathcal C\to \mathcal C$, called the \emph{shift functor} and
    \item a collection of \emph{triangles}, which are triples of objects and morphisms written as 
    \[A\xrightarrow{f} B \xrightarrow{g} C\xrightarrow{h} \Sigma A.\]
\end{itemize}
Denote by $X[n]=\Sigma^nX$.
This data is required to satisfy the axioms for a triangulated category,
\begin{description}
    \item[TR1], concerning which triangles must exist:
    \begin{itemize} 
        \item The triangle $X\xrightarrow{\id} X\to 0 \to \Sigma X$ is an exact triangle
        \item  For every morphism $f:X\to Y$ there exists an object (called the cone) so that $X\to Y \to \cone(f)$ is an exact triangle
        \item Every triangle which is isomorphic to an exact triangle is exact.
    \end{itemize}
    \item[TR2], concerning the interchange between exact triangles and suspension. If $X\xrightarrow{f} Y \xrightarrow{g} Z\xrightarrow{h} X[1]$ is an exact triangle, then so are $Y\to Z\to X[1]\to Y[1]$ and $Z[-1]\to X\to Y\to Z$.
    \item[TR3] Given a commutative square, if we complete the rows to exact triangles, then there exists a morphism between the third objects making everything commute.
    \item[TR4] The octahedral axiom, which states that given exact triangles
    \begin{align*}
        X\xrightarrow{f} Y \xrightarrow{g} Z'\xrightarrow{h} X[1]\\
        Y\xrightarrow{i} Z \xrightarrow{j} X'\xrightarrow{k} Y[1]\\
        X\xrightarrow{i\circ f} Z \xrightarrow{l} Y'\xrightarrow{m} Z[1]
    \end{align*}
    There exists a triangle $Z'\to Y'\to X'\to Z'[1]$.
    making the diagram of these triangles commute.
\end{description}
 


 
%label:"def:mappingConeOfCochainComplexes"
%type:"definition"
%author:JeffHicks
%name:"mapping cone of cochain complexes"
%parent:"0000"

 
 
Given two cochain complexes $M^\bullet$ and $N^\bullet$ and a chain map $f: M^\bullet\to N^\bullet$, the \emph{cone of $f$} is a new chain complex
\[\text{cone}(f):=\left(M^{i+1}\oplus N^i, d=\begin{pmatrix}- d_M^{i+1} & 0 \\ - f^{i+1} & d_N^i\end{pmatrix}\right).\]

 
We observe that when $X, Y$ are simplicial spaces and $f:X\to Y$ is a simplicial map that simplicial cochains topological mapping cone (\cref{def:topologicalMappingCone}) are the cone-cochains,
\[C^\bullet(\cone(f))=\cone(f^*:C^\bullet(X)\to C^\bullet(Y)).\]
This does not end up giving a triangulated category, as it does not satisfy \todo{example}
However, the category of chain complexes with morphisms modulo chain homotopies is a triangulated category. This is called the \emph{homotopy category}.
Triangulated categories capture many important aspects of homological algebra for chain complexes through the study of cohomological functors. 
%label:"def:cohomologicalFunctor"
%author:JeffHicks
%name:"cohomological Functor"
%type:"definition"

 
 
Let $\mathcal C$ be a triangulated category, and $\mathcal A$ be an abelian category.
A \emph{cohomological functor} is a functor $F: \mathcal C\to \mathcal A$ which sends exact triangles 
\[A\to B\to C\to A[1]\]
to exact sequences
\[F(A)\to F(B)\to F(C).\]

 
From this, we see that cohomological functors associate to exact triangles in $\mathcal C$ a \emph{long exact sequence}
\[\cdots F(C[-1])\to F(A)\to F(B)\to F(C)\to F(A[1])\to \cdots \]
%tag:000B
%label:"prp:homologyIsCohomologicalFunctor"
%author:JeffHicks
%name:"cohomology is a cohomological functor"
%type:"proposition"
%proof:prf:homologyIsCohomologicalFunctor

 
 
Let $\mathcal A$ be an abelian category, and $\mathcal K(\mathcal A)$ the homotopy category, the functor 
\[H^0: \mathcal K(\mathcal A)\to \mathcal A\]
is an example of a cohomological functor.

 

%parent:prp:homologyIsCohomologicalFunctor
%author:JeffHicks
%name:"cohomology is a cohomological functor"
%type:"proof"
%label:"prf:homologyIsCohomologicalFunctor"

 
    The idea of proof is to observe that every exact triangle in the homotopy category is homotopic to one of the form $A\xrightarrow{f} B \to \cone(f)$, which is an exact sequence of chain complexes. The long exact sequence of cohomology groups arising from the snake lemma then proves the proposition.
 
