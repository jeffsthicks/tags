%tag:0007
%label:def:heegaardMoves
%author:JeffHicks
%name:"Heegaard moves"
%type:definition


    Let $(\Sigma_g,\{\alpha_i\}_{i=1}^g, \{\beta_i\}_{i=1}^g)$ be a Heegaard diagram. We say that another diagram $(\Sigma_g,\{\alpha_i'\}_{i=1}^{g}, \{\beta_i'\}_{i=1}^{g})$ is related to $(\Sigma_g,\{\alpha_i\}_{i=1}^g, \{\beta_i\}_{i=1}^g)$ by an
    \begin{itemize}
        \item isotopy if the sets $\{\alpha_i\}_{i=1}^g, \{\alpha_i'\}^g$ are isotopic in $\Sigma_g$, or  $\{\beta_i\}_{i=1}^g, \{\beta_i'\}^g$ are isotopic in $\Sigma_g$.
        \item handle slide if $\alpha_i=\alpha_{i'}$ for $i\neq g$, and the curves $\alpha_{g-1}, \alpha_g, \alpha_g'$ bound a pair of pants in $\Sigma_g$ disjoint from $\{\alpha_i\}_{i=1}^{g-2}$ (or similarly for the $\beta_i$). See \cref{fig:handleslide}
    \end{itemize}

