%tag:0007
%label:"art:geometryOnVectorSpaces"
%author:JeffHicks
%name:"Geometry on Manifolds "
%type:"article"

In the broadest sense, differential geometry is the study of smooth manifolds equipped with additional data fixing some ``geometry'' on the space. 
Usually, this geometry is obtained by imposing some geometry of vector spaces onto the tangent space of a smooth manifold.
\begin{example}
    An inner product on a vector space $g: V\times V\to \RR$ determines a geometry on the vector space by specifying lengths and angles.
    A \emph{Riemannian manifold} is a pair $(X, g)$ where $X$ is a smooth manifold and $g$ denotes a family of inner products 
    \[g_p: T_pX\to T_pX\]
    which vary smoothly on the parameter in $p\in X$. 
    A Riemannian manifold has local notions of lengths and angles that determine geodesics, curvature, and volumes.
    Many constructions in Riemannian geometry are determined locally, as there are no global obstructions to locally modifying the Riemannian metric.
    However, Riemannian geometry posses some \emph{local invariants,} which means that neighborhoods of points are distinguishable from each other using invariants such a curvature.
\end{example} 
%label:"def:complexVectorSpace"
%author:JeffHicks
%name:"complex vector space"
%type:"definition"

 
 A complex space is a vector space $V$ of dimension $2n$, along with a map of vector bundles $J: V\to V$ with $J^2=-\operatorname{id}_V$. 

 
%label:"def:almostComplexManifold"
%author:JeffHicks
%name:"almost complex manifold"
%type:"definition"

 
 
An \emph{almost complex manifold} is a pair $(X,J)$ where $X$ is a $2n$-dimensional manifold equipped with a bundle morphism (called the \emph{almost complex structure})
\[J: TX\to TX\]
so at each fiber $(T_xX, J_x)$ is a \snip{complex vector space}{def:complexVectorSpace}.

 
Symplectic geometry, when taken outside of its historical context, seems a bit artificially constructed. 
%tag:0002
%label:"def:symplecticVectorSpace"
%author:JeffHicks
%name:"symplectic vector space"
%type:"definition"

 
 A \emph{symplectic vector space} is a $2n$-dimensional vector space $V$, along with a choice of symplectic form $\omega: V\times V\to \RR$ which is an antisymmetric and satisfies any of the following equivalent non-degeneracy conditions:
\begin{itemize}
    \item $v \in V$ is the zero vector if and only if $\iota_v\omega=0$,
    \item The symplectic form gives an isomorphism between $V$ and its dual:
    \begin{align*}
        \omega:V\to V^* &&
        v\mapsto \iota_v\omega,
    \end{align*}
    \item The top form $\omega^{n}$ is non-zero.
\end{itemize}

 
%tag:0003
%label:def:symplecticManifold
%author:JeffHicks
%name:"symplectic manifold"
%type:definition

 
 A \emph{symplectic manifold} is a pair $(X,  \omega)$ where $X$ is a smooth manifold of dimension $2n$ equipped with a \emph{symplectic form} 
\[\omega\in \Omega^2(X; \RR).\]
which is closed (i.e. \(d\omega=0\)) and at each point $x\in X$ makes the pair $(T_xX, \omega_x)$ a \sref{def:symplecticVectorSpace}.

 
If Riemannian geometry yields a theory of lengths and complex geometry a theory of preferred right-angles, then symplectic geometry gives a theory of signed areas. 
There is a certain notion of compatibility between these three kinds of geometry. For example, notice that an inner product also identifies right angles and areas. 
\input{def_compatibleAlmostComplexStructure}
Remarkably, having a symplectic structure is sufficient for the construction of a compatible  almost complex structure. 
%label:"prp:compatibleACS"
%author:JeffHicks
%name:"Existence of compatible ACS"
%type:"proposition"
%Source:"Proposition  12.6, \cite{da2001lectures}"

 
 
Every symplectic manifold has a compatible almost-complex structure. 

 