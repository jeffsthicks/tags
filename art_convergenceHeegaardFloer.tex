%tag:0007
%label:"art:convergenceHeegaardFloer"
%author:JeffHicks
%name:"admissibility and convergence"
%type:"exposition"

In order to obtain a replacement for the symplectic form providing a bound on the energy of holomorphic strips, we need to work a little bit harder. The key insight is that there is a dictionary between holomorphic strips in the symmetric product and maps from more-complicated domains to the surface.

%label:prp:intersectionsInSymmetricProducts
%author:JeffHicks
%name:"intersection of Lagrangians in symmetric product"
%type:proposition


    Let $\underline \alpha = \{\alpha_i\}_{i=1}^g, \underline \beta = \{\beta_i\}_{i=1}^g$ be $g$-tuples of disjoint curves in $\Sigma_g$, so that $\alpha_i\cap \beta_j$ intersect transversely. Then there is a bijection between
    \[L_{\underline \alpha}\cap L_{\underline \beta} = \{(x_1, \ldots, x_g)\;|\; x_i\neq x_j, x_i \in \underline \alpha \cap \underline \beta.\}\]

This means that we can understand the intersections between our Lagrangians $L_{\underline \alpha}, L_{\underline \beta}$ in terms of the intersections between the collection of cycles. Remarkably, we can also understand the holomorphic strips between the intersections points by looking at data on $\Sigma_g$. 
%label:"thm:stripsInSymmetricProducts"
%author:JeffHicks
%name:"strips in the symmetric product"
%type:"theorem"


    Given a holomorphic strip $u\in \mathcal M(x, y)$ there exists a $g$-branched cover
    $\pi: \hat D\to D$ 
    and a holomorphic map from $\hat u: \hat D \to \Sigma_g$ so that for all $z\in D$, 
    \[u(z)=[(\hat u(z_1), \hat u(z_2), \cdots , \hat u(z_g)])\]
    where $\{z_1, \ldots, z_g\}\in \pi^{-1}(z)$.

With this relation, we can ``by hand'' rule out the kinds of problematic disks which would interfere with the arguments of Gromov-compactness.
%label:"def:heegaardDomain"
%author:JeffHicks
%name:"Heegaard domain"
%type:"definition"


    A \emph{Heegaard domain} (or simply domain) is a formal linear combination of the connected components of $\Sigma\setminus (\underline \alpha\cup \underline \beta)$.

%label:"def:periodicHeegaardDomain"
%author:JeffHicks
%name:"periodic Heegaard domain"
%type:"definition"

    A domain is \emph{periodic} if its boundary can be written as a sum of the cycles in $\underline \alpha, \underline \beta$ and it has no intersection with the marked point $z$.

    Every periodic domain can be represented by a surface with boundary $\hat D\to \Sigma$, thus every periodic domain $\mathcal D$ gives a homology class $H_2(M; \ZZ)$ by gluing the attaching disks associated to each of the $\alpha_i, \beta_j$ to the appropriate boundaries in $\hat D$. We call this homology class $H(\mathcal D)$.
%label:def:weaklyAdmissibleHeegaardDiagram
%author:JeffHicks
%name:"weakly admissible Heegaard diagram"
%type:definition


    A pointed Heegaard diagram is \emph{weakly admissible} if for a spin-c structure $s$ if for every non-trivial periodic domain $\mathcal D$ with 
    \[\langle c_1(s), H(\mathcal D)\rangle = 0\]
    $\mathcal D$ has both positive and negative coefficients.

The following lemma may give us some intuition for where the admissibility condition enters into the definition of Heegaard-Floer cohomology.
%label:"lem:heegaardAdmissibleZeroArea"
%author:JeffHicks
%name:"zero area periodic domains"
%type:"lemma"
%source:ozsvath2004holomorphic
%sourceDetail:"lemma 4.12"

    The following are equivalent:
    \begin{itemize}
        \item $(\Sigma_g, \underline \alpha, \underline \beta)$ is admissible for all $\Spinc$ structures
        \item There exists a symplectic form on $\Sigma$ so that every periodic domain as total signed area 0. 
    \end{itemize}

%label:"lem:heegaardAdmissibleFiniteStrips"
%author:JeffHicks
%name:"finitely many strips in Heegaard differential"
%type:"lemma"
%source:"ozsvath2004holomorphic"
%sourceDetail:"Lemma 4.14"


    Suppose that $(\Sigma, \alpha, \beta, z)$ is a weakly admissible Heegaard diagram. There are only finite many $\phi\in \pi_2(x, y)$ with $\mu(\phi)-j, n_z(\phi)=k, \mathcal D(\phi)\geq 0$.

This shows that the symplectic energy of the holomorphic strips that we consider in the definition of the differential $d_z$ is bounded. 