%tag:0007
%label:art:maximumPrinciples
%author:JeffHicks
%name:"maximum principles"
%type:exposition
We recall some of the basics of Hamiltonian Floer theory. Let $X$ be a compact symplectic manifold. 
Given $H_t: X\times \RR\to \RR$ a time dependent Hamiltonian, we obtain an action on the free loop space
%type:equation
%label:eqn:floerAction
%name:"Floer action"


\begin{equation}
    A_{H_t}(\gamma)=-\int_\gamma \lambda + \int_\gamma H_t dt
    \label{eqn:floerAction}
\end{equation}
whose critical points are the time one periodic orbits of $V_{H_t}$. Given a $\omega$-compatible almost complex structure, we observe that cylinders $u: \RR_s\times S^1_t\to X$ which satisfy the $H_t$ perturbed Floer equation
%type:equation
%label:eqn:floerEquation
%name:"Floer equation"


\begin{equation}
    \partial_s u + J(\partial_t u - V_{H_t})=0
    \label{eqn:floerEquation}
\end{equation}
parameterize the negative gradient flow lines of $A_{H_t}$. For curves $\gamma_+, \gamma_-$, let $\mathcal M(\gamma_+, \gamma_-)$ denote the moduli space of solutions to Floer's equation with ends limiting to $\gamma_+, \gamma_-$.
Supposing that $X$ is an exact symplectic manifold, and that our time-dependent Hamiltonian is chosen in a generically, the Floer cochains $\CF(X, H_t)$ are the graded vector space generated  on the time one orbits of $H_t$.
We take a slightly different convention (following Equation 5.2 of \cite{abouzaid2010geometric}) and give each orbit the grading $\deg(\gamma)=n-CZ(\gamma)$, where $CZ(\gamma)$ is the Conley-Zehnder index.
The structure coefficients of the differential are given by counts of solutions to Floer's equation. The theory becomes powerful when it satisfies the following properties:
\begin{itemize}
    \item $\CF(X, H_t)$ is a chain complex. The key step is to show that when $\dim(\mathcal M(\gamma_+, \gamma_-))=1$, there exists a compactifiaction of this moduli space by including broken cylinders. A compactification of the space comes from applying Gromov compactness, while additional requirements on $X$ are sometimes required ensure that the only configurations which appear in the compactification are broken cylinders. In our setting, the only breaking configurations which may occur are broken cylinders, as $X$ is exact (so $\omega(\pi_2(X))=0)$.
    \item Given $H_{t, 0}$ and $H_{t, 1}$ two time-dependent Hamiltonians, there exists a continuation map $\CF(X, H_{t, 0})\to \CF(X, H_{t, 1})$. Furthermore, this map is a homotopy equivalence. 
    \item Finally, we need some way to compute $\CF(X, H_{t, 1})$. One way to do this is to observe that for $C^2$ small Hamiltonians the Floer cochains agree with the Morse cochains (and only consist of constant orbits). By either using the PSS-isomorphism or by analyzing Floer trajectores, the Floer cohomology can be compared to the Morse cohomology of $X$.
\end{itemize}
The major difference in defining the Hamiltonian Floer ccohomology for Liouville domains $X$ (as opposed to compact symplectic manifolds) comes from the proof of Gromov-compactness. The first step in the proof of Gromov-compactness is to apply Arzel\'a-Ascoli to out sequence of psuedoholomorphic maps.  Because $\hat X$ is not compact, we cannot apply the Arzel\'a-Ascoli theorem to a sequence of pseudoholomorphic cylinders $u: S^1\times \RR \to \hat X$. 

We now give an example of where we can solve the issue of non-compactness.
%label:"exm:compactnessFromMaximumModulus"
%name:"maximum modulus provides compactness"
%type:"example"



    The maximum modulus principle states that if $\phi: D^2\to \CC$ is a holomorphic function from the disk to $\CC$, that the maximum of $|\phi|: D^2\to \RR_{\geq 0}$ is achieved on $\partial D^2$.

    Let $\hat X$ be a non-compact symplectic manifold with compatible almost complex structure $J$, along with a $J-\jmath$-holomorphic projection $W: \hat X\to \CC$. Suppose that the fibers of $W$ are compact. Pick two loops $\gamma_-, \gamma_+\subset \hat X$ and $r_0\in \RR$ large enough so that $U:=W^{-1}(\{z \st |z|\leq r\})$ contains $\gamma_-, \gamma_+$. We will prove that every pseudoholomorphic cylinder $u: S^1\times \RR\to \hat X$ with ends limiting to $\gamma_-, \gamma_+$ has image contained within the compact subset $U$. 

    The composition $W\circ u: S^1\times \RR\to \CC$ is a holomorphic map, with ends limiting to $W(\gamma_\pm)$, and therefore satisfies the maximum modulus principle. Since the boundary is sent to $W(\gamma_\pm)$, we obtain that $|W|$ achieves a value no greater than $r_0$ on $u$; therefore $\Im(u)\subset U$. It follows that the image of $u$ is contained within a compact set. 
    \label{exm:compactnessFromMaximumModulus}

In order to extend \cref{exm:compactnessFromMaximumModulus} to the setting of $\hat X$, we will use the maximum principle. First, we will need to assume that we have chosen our almost complex structure for $\hat X$ so that the sub-bundle spanned by the vector fields $\partial_r, R$ form an almost complex subspace.
%tag:000K
%label:def:contactTypeACS
%author:JeffHicks
%name:"contact type almost complex structure"
%type:definition

 
    Let $\hat X$ be the completion of a Liouville domain. A choice of almost complex structure for $\hat X$ is \emph{of contact type} if 
    \[d(\exp(r))\circ J = -\alpha.\]
    \label{def:contactTypeACS}
 
We will also need to assume that we have choosen our Hamiltonian so that over the symplectization it only depends on the $r$-coordinate.
For such a contact type almost complex structure and Hamiltonian there exists a verion of \cref{exm:compactnessFromMaximumModulus}.
%label:"prp:liouvilleIsGeometricallyBounded"
%author:JeffHicks
%name:"Liouville manifolds are geometrically bounded"
%type:"proposition"

 
   \label{prp:liouvilleIsGeometricallyBounded}
    Let $H: \hat X\to \RR$ be a Hamiltonian which on the symplectization takes the form of $h(\exp(r))$. 
    Let $\gamma_+, \gamma_-$ be time 1 orbits of $V_{H_{t}}$. For a contact type almost complex structure, every solution $u: \RR\times S^1\to \hat X$ of the Floer equation with ends limiting to $\gamma_+, \gamma_-$ has image contained in the subset $\hat X|_{\exp(r)\leq C}$, where $C$ is the maximum value of $\exp(r)$ on the orbits $\gamma_+, \gamma_-$.
 
%label:"prf:liouvilleIsGeometricallyBounded"
%author:JeffHicks
%name:"Liouville manifolds are geometrically bounded"
%type:"proof"
%parent:prp:liouvilleIsGeometricallyBounded
%source:"seidel2006biased"

 
      Let $u: \RR\times S^1\to \RR$ be a solution to the Floer equation (\cref{eqn:floerEquation}). Let $\rho=\exp(r\circ u)$.  By applying $d(\exp(r))$ to the Floer equation, and using \cref{def:contactTypeACS} we obtain :
      \begin{align*}
         0=d(\exp(r))\circ \left(\partial_s u + J(\partial_t u - V_{H})\right)=& \partial_s(\rho) - \alpha(\partial_t u)+ \alpha(V_H)
      \end{align*}
      Because $H= h(\rho)$, the Hamiltonian vector field associated to $H$ is $h'(\rho) V_\alpha$, where $V_\alpha$ is the Reeb flow. From \cref{rem:increasingHamiltonian}, we see that $\alpha(V_H)=h'(\rho).$
      \begin{align*}
         =& \partial_s(\rho) - \alpha(\partial_t u)+ h'(\rho) .
      \end{align*}
      Similarly, applying $\alpha$ to the Floer equation:
      \begin{align*}
         0=\alpha  \left(\partial_s u + J(\partial_t u - V_{H})\right) =& \alpha(\partial_su)+ \partial_t(\exp(\rho))+ V_{H}(\exp(\rho))
         \intertext{Because $H= h(\rho)$ has the same level sets as $\rho$,  $V_H(\rho)=0$. }
         =& \alpha(\partial_s u) + \partial_t(\rho)
      \end{align*}
      Differentiating the first line with respect to $s$,  the second line with respect to $t$, and summing the lines together  we obtain 
      \begin{align*}
        0=&  (\partial_s^2 + \partial_t^2)\circ \rho- \partial_s\alpha(\partial_t u)+\partial_s \rho h'(\rho) +\partial_t\alpha(\partial_s u)\\
        \intertext{As $[\partial_s, \partial_t]=0$, we can substitue $-\partial_t\alpha(\partial_s u)+ \partial_s\alpha (\partial_t u)= -u^*\omega(\partial_s, \partial_t)$ }
        =& \Delta \rho- u^*\omega(\partial_s, \partial_t)+ \rho h'(\rho)\partial_s\rho + \rho h''(\rho) \partial_s\rho\\
        \intertext{By again applying Floer's equation, and using the compatibility of almost complex structure with $J$, we may substitue $u^*\omega(\partial_s, \partial_t)= u^*\omega(\partial_s, J\partial_t-X_H)=|\partial_s u^2|-dh'(\rho)\partial_s$ }
        =& \Delta \rho-|\partial_s|^2+\rho h''(\rho)\partial_s\rho 
      \end{align*}
      We therefore obtain that $\Delta\rho+\rho\cdot h''(\rho) \partial_s\rho\geq 0$. 

      Observe now that where $z\in S^1\times \RR$ is a proposed maximum for $\rho$ that $\partial_s\rho=0$, allowing us to write $(\partial^2_s \rho + \partial^2_t \rho)|_z \geq 0$. This implies that at least one of $\partial^2_s, \partial^2_s$ has to be non-negative --- in particular, the second derivative test does not detect the maximum. A more general argument --- the maximum principle --- states that $\rho$ achieves no local maxima; therefore $\sup_{S^1\times \RR} \rho \leq \max_{t\in S^1} \exp(r\circ \gamma_\pm)=:C.$ It follows that the image of $u$ is contained in $\hat X|_{\rho< C}$, which is a compact set.
 
