%label:"exm:reebVectorField"
%author:JeffHicks
%name:"contact hypersurfaces"
%type:"example"
%parent:"exm:contactManifold,exm:reebVectorField"


    We return to \cref{exm:contactManifold} of hypersurfaces in $\RR^{2n}$ given by $M=p^{-1}(N)$. 
    Consider the hypersurface $N$ defined by the equation $p_1+\cdots p_n=1$. Then $M=S^{2n-1}\subset \RR^n$.
    We give $\CC^n$ the polar coordinates $(r_i, \theta_i)$. 

    Let $f=\sum_{i=1}^n |z_i|^2.$ 
    The tangent space to $M$ is the orthogonal complement to $\grad(f).$ Consider the vector field 
    \[V:=\sum_{i=1}^n  x_i \partial_{y_i}- y_i \partial_{x_i}.\]
    First, observe that 
    \[V\cdot \grad(f)=\sum_{i=1}^n( x_i y_i - y_i x_i )= 0 \]
    so $V$ restricts to a vector field on $M$. 
    Let $v=\sum_{i}a_i \partial_{x_i}+b_i\partial_{y_i}$ be any vector in  $TM$. Then the pairing
    \begin{align*}
        \omega(v, V)=&\sum_{i=1}^n a_ix_i+b_iy_i\\
        =&v\cdot \grad(f)=0
    \end{align*}
    From this, we conclude that $V\in \ker(d\alpha)$.
    Finally, we have that $\alpha=\iota_Z\omega$, so 
    \[\omega(Z, V)=\sum_{i=1}^n (x^2+y^2)=1\]
    From which we conclude that $V$ is the Reeb vector field for $(M, \alpha)$.

