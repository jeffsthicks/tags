%label:"art:mosersTheorem"
%author:JeffHicks
%name:"local models for symplectic manifolds"
%type:"article"

In contrast to Riemannian geometry,  symplectic manifolds are ``locally symplectomorphic.''
This means that one cannot distinguish two symplectic manifolds simply based on the symplectic geometry of a small neighborhood of a point. 
In this section, we sketch a proof of this fact.
%label:"moser1"
%type:"theorem"
%name:"Mosers Theorem"


    Let $V_0$ and $V_1$ be closed connected $n$ manifolds with volume forms $\Omega_0$ and $\Omega_1$.
     Suppose that they have the same total volume,  that is 
    \[\int_{V_0} \Omega_0 = \int_{V_1} \Omega_1.\]
    Let $\phi_0: V_0\to V_1$ be a diffeomorphism.
    Then $\phi_0$ is isotopic to a volume preserving diffeomorphism $\phi_1$; that is
    \[\phi_1^*\Omega_1=\Omega_0.\]



\begin{proof}
    Without loss of generality,  let $V_0=V_1$ and let $\phi_0$ be the identity. 
    Let $\Omega_t=(1-t)\Omega_0+t\Omega_1$. 
    We want to find $\{\phi_t\}$ with $\phi_t^*\Omega_t=\Omega_0$. 
    We could equivalently describe such an isotopy $\phi_t$ by the vector field
    \[V_t:=\left(\frac{d}{dt}\phi_t\right)\circ\phi_t.\]
    Taking Lie derivatives,  we have the pullback condition is equivalent to  
    \[\frac{d}{dt}(\phi_t^*\Omega_t)=\phi_t^*\left(\mathcal L_{V_t} \Omega_t+\frac{d}{dt}\Omega_t\right)\]
    So we are looking for $V_t$ with $\mathcal L_{V_t}\Omega_t+\frac{d}{dt}\Omega_t=0$.
    Simplifying further gives us 
    \[
        d\iota_{V_t}\Omega_t=-\frac{d}{dt}\Omega_t=\Omega_0-\Omega_1\]
        By Stoke's theorem, the right term is exact as $\int_X\Omega_0=\int_X \Omega_1$. Therefore there exists some $n-1$ form $\eta$ so that
        $d\iota_{V_t}\Omega_t=d\eta$.
    So now we have to solve $\iota_{V_t}\Omega_t=\eta$. 
    Since $\Omega_t$ is a volume form,  then there exists a $V_t$ satisfying this equation. 
\end{proof}
This tells us that symplectomorphisms of surfaces are boring because any diffeomorphism can be made into a symplectomorphism. 
In fact, every isotopy of symplectic forms in the same cohomology can be realized by a family of symplectomorphisms. 
%label:"moser2"
%type:"theorem"
%name:"Mosers Theorem"


    Let $X$ be a closed $(2n)$-manifold.
    Let $\{\omega_t\}$ be a smooth family of symplectic forms in the same cohomology class.
    Then there exists a smooth family of diffeomorphism $\{\phi_t\}$ with $\phi_0=\operatorname{id}_X$ and $\phi_t^*\omega_t=\omega_0$.


\begin{proof}
    Instead of finding $\phi_t$, we instead search for the vector $V_t$ generating the isotopy.
    Because the symplectic form should be invariant under this isotopy, we obtain the following condition on $V_t$:
    \begin{align*}  0=&\frac{d}{dt}(\phi_t^*\omega_t)=\phi_t^*\left(\mathcal L_{V_t}\omega_t+\frac{d}{dt}\omega_t\right)\\
        =&\mathcal L_{V_t}\omega_t + \frac{d}{dt}\omega_t\\
        =&d\iota_{V_t}\omega+\frac{d}{dt}\omega_t
    \end{align*}
    Since the cohomology class $[\omega_t]$ is constant,  the time derivative  $\frac{d}{dt}\omega_t$ is exact. Therefore there exists \footnote{Small technical remark: We need $\eta_t$ to depend smoothly on $t$, which can be achieved by using the Hodge decomposition.}  $\eta_t$ with $d\eta_t=\frac{d}{dt}\omega_t$.
        This reduces our previous computation to
    \begin{align*}
        =&d\iota_{V_t}\omega_t+d\eta_t
    \end{align*}
    Since $\omega_t$ is nondegenerate,  there exists unique $V_t$ with $\iota_{V_t}\omega_t+\eta_t=0$. 
\end{proof}

We also have a relative version of the Moser theorem.
%label:"thm:relativemoser"
%type:"theorem"
%name:"relativeMoser"


    Let $X$ be a manifold. Let $Y$ be a compact submanifold of $X$. 
    Let $\omega_0,  \omega_1$ be symplectic forms on $Y$ such that $\omega_0(p)=\omega_1(p)$ for all $p\in X$.
    Then there exists neighborhoods $U_0,  U_1\supset X$ and a symplectomorphism $\phi:(U_0,  \omega_0)\to (U_1,  \omega_1)$ such that $\phi(p)=p$ for all $p\in X$.
\begin{proof}
    Let $\omega_t=(1-t)\omega_0+t\omega_1$ for $t\in [0, 1)$. Then $\omega_t$ is symplectic for all $t$ in a sufficiently small tubular neighborhood $N$ of $X$. We want to find $\phi_t$ with $\phi_t^*\omega_t=\omega_0$ for all $t$. Let $V_t=\frac{d}{dt}\phi_t\circ \phi_t$. Recall that 
    \begin{align*}
        \frac{d}{dt}(\phi_t^*\omega_t)=&\phi^*_t(\mathcal L_{V_t}\omega_t+(\omega_1-\omega_0))\\
        =&\phi^*_t(di_{V_t}\omega_t+(\omega_1-\omega_0))
    \end{align*}
        Since $i^*\omega_0=i^*\omega_1$,  we have that $[\omega_0]=[\omega_1]\in H^2(N)$ . Then there exists $\eta$  such that $d\eta=\omega_0-\omega_1$ and $\eta(p)=0$ for all $p\in X$. 
    \begin{align*}
        =&\phi^*d(i_{V_t}\omega-\eta)
    \end{align*}
    Since $\omega_t$ is nondegenerate,  there exists unique vector field $V_t$ with $i_{V_t}\omega_t=\eta$. Since $\eta(p)=0$ for all $p\in X$,  it follows that $V_t(p)=0$ for all $p\in X$.
    For this $V_t$ we have that $\frac{d}{dt}(\phi^*_t\omega_t)=0$. Therefore $V_t$ generates a family of diffeomorphism $\{\phi_t\}$ which are the identity on $X$ and defined on a neighborhood of $X$.
\end{proof}
The application of this theorem to symplectic geometry shows that there are no local invariants of symplectic manifolds. 
%author:JeffHicks
%name:"Darboux's theorem"
%type:"corollary"
%label:"cor:darbouxsTheorem"   

If $(X,  \omega_0)$ is symplectic then for all $p\in X$ there exists neighborhoods $U_0\supset \{p\}$ and $U_1\subset \RR^{2n}$ and a symplectomorphism $(U_0,  \omega_0)\to (U_1,  \omega_{std})$.