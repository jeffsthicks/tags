%label:"art:B-sideinvariants"
%type:"article"
%name:"B-side invariants"
%caption:""


\input{def_dgcategoryofcoherentsheaves}

\input{def_semi-orthogonalDecomposition}

    When $X$ is Fano, typically $D\Coh(X)$ admits a semi-orthogonal decomposition \cref{def:SemiOrthogonalDecomposition}
    \[D\Coh(X) = \langle \C_1, \cdots \mathcal C_n\rangle.\]  


\input{exm_Beilinsoncollection}

\input{exm_Avarietywithnofullexceptionalcollection}
When we don't have a full exceptional collection, we can try ``as hard as we can'' to build one, and look at what is left. Typically, we have a semi-orthogonal decomposition 
\[D\Coh(X) = \langle \text{Ku}(X), \mathcal C_1, \ldots, \mathcal C_n\rangle\]
where $\mathcal C_i = D\Coh(\bullet)$, and a bigger part $\text{Ku}(X)$, which is often called the \emph{Kuznetsov component} of $D\Coh(X)$. 
Often this Kuznetsov component contains information about the birational geometry of $X$. They also reveal hidden relationships, e.g. it could be that $D\Coh(X)\neq D\Coh(X')$, but we have $\text{Ku}(X) = \text{K}(X')$. 


