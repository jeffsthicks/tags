%label:"art:BSideInvariants"
%type:"article"
%name:"B-side invariants"
%caption:""
%parent:"art_HMSForFanos2"


%label:"def:DgCategoryOfCoherentSheaves"
%type:"definition"
%name:"dg category of coherent sheaves"
%caption:""
%parent:"art_BSideInvariants"


Let $X$ be an algebraic variety. Denote by $D\Coh(X)$ the dg-enhancement of the bounded derived category of coherent sheaves on $X$. 



%label:"def:SemiOrthogonalDecomposition"
%type:"definition"
%name:"semi-orthogonal Decomposition"
%caption:""
%parent:"art_BSideInvariants"


Let $\mathcal{C}$ be a triangulated category. A \emph{semi-orthogonal decomposition} of $\mathcal{C}$ is the data 
\[\mathcal{C} = \langle \mathcal{C}_0, \ldots, \mathcal{C}_i\rangle\]
where 
\begin{itemize}
    \item $\mathcal{C}_i$ are full triangulated subcategories
    \item $\hom(\mathcal{C}_i, \mathcal{C}_j) = 0$ whenever $C_i \in \mathcal{C}_i, C_j \in \mathcal{C}_j$ and $j > i$ 
    \item The $\mathcal{C}_i$ generate the category. 
\end{itemize} 
 When $\mathcal{C}_i = D\Coh(\bullet) = D\Vect$, we say that $\mathcal{C}$ admits a full exceptional collection. 



When $X$ is Fano, typically $D\Coh(X)$ admits a semi-orthogonal decomposition \cref{def:SemiOrthogonalDecomposition}
    \[D\Coh(X) = \langle \mathcal{C}_1, \cdots \mathcal{C}_n\rangle\]  

%label:"exm:BeilinsonCollection"
%type:"example"
%name:"Beilinson collection"
%caption:""
%parent:"art_BSideInvariants"


When $X = \mathbb{P}^n$ or a toric Fano variety,  then $D^b\Coh(X)$ admits a full exceptional collection. 



%label:"exm:AVarietyWithNoFullExceptionalCollection"
%type:"example"
%name:"A variety with no full exceptional collection"
%caption:""
%parent:"art_BSideInvariants"


 It's easy to find varieties without a full exceptional collection. Suppose that $X$ admits a full exceptional collection. Then $H^{p, q}(X) = 0$ for $p \neq q$.



When we don't have a full exceptional collection, we can try to build one as best as we can and look at what is left. Typically, we have a semi-orthogonal decomposition 
\[D\Coh(X) = \langle \text{Ku}(X), \mathcal{C}_1, \ldots, \mathcal{C}_n\rangle\]
where $\mathcal{C}_i = D\Coh(\bullet)$, and a larger part $\text{Ku}(X)$, which is often called the \emph{Kuznetsov component} of $D\Coh(X)$. 
Often, this Kuznetsov component contains information about the birational geometry of $X$. It can also reveal hidden relationships, e.g., it may be that $D\Coh(X) \neq D\Coh(X')$, but we have $\text{Ku}(X) = \text{Ku}(X')$. 


