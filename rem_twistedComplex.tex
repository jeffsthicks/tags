%tag:000R
%label:rem:twistedComplex
%author:JeffHicks
%name:"properties of twisted complexes"
%type:remark

 


    One viewpoint on twisted complexes is that they give \emph{deformations} of (direct sums of) objects of our categories. In the simplest case --- the mapping cone of chain map $f: A\to B$, the differential on $\cone(f)$ has the form 
    \[d_{\cone(f)}=\begin{pmatrix} d_A & 0 \\ 0 & d_B \end{pmatrix} + \begin{pmatrix} 0 & 0\\ f & 0\end{pmatrix}\]
    where the first term is the differential on $A\oplus B[1]$, and the second term ```deforms'' the differential on this chain complex.

    Twisted categories extend this story in several directions: firstly, we expand the set of deformations so that the objects we consider are chain complexes up to homotopy, and we allow deformations of the product (and not only differential) structure.

    With regards to the first point: Suppose that we have a (not necessarily exact) sequence of chain complexes $A\xrightarrow{f} B \xrightarrow{g} C$. The total complex of this sequence will not be a chain complex (as $g\circ f \neq 0$). However, to build a twisted complex from this data we will only need that $g\circ f$ is homotopic to zero. Suppose that  $H:A\to C[1]$ is a homotopy (so that $d_AH+Hd_C=g\circ f$). Then  
    \[\delta = \begin{pmatrix}
        0 & 0 & 0\\
        f & 0 & 0\\
        H & g & 0
    \end{pmatrix}\]
    gives us the structure of a twisted complex on $A\oplus B[1]\oplus C[2]$.

    For the second point: Let $(A, m^k)$ be an $A_\infty$ algebra. There are a particularly nice class of deformations of $A_\infty$ governed by elements $a\in A^1$ satisfying the Maurer-Cartan equation:
    \[m^1(a)+m^2(a\otimes a)+m^3(a\otimes a \otimes a)+\cdots =0.\]
    In order for this equation to make sense, one needs show that the sum converges. This is usually achieved by asking that $A$ be filtered and that $m^k(a^{\otimes k})$ lies increasingly positive filtration levels.
    When one can make sense of this equation, we can define a new $A_\infty$ algebra, $(A, m^k_a)$ whose product is defined by 
    \[m^k_a:=\sum_{n>0}\sum_{j_0+\cdots+j_k=n} m^{k+n}(a^{\otimes j_0}\otimes \id \otimes a^{\otimes j_1}\otimes \id \cdots \otimes a^{\otimes j_{k-1}}\otimes \id \otimes a^{\otimes j_k})\]

    Now consider the setting where $C$ is a chain complex, and $A=\hom(C, C)$. Then $a\in A^1$ corresponds to a map $a: C\to C[1]$, and the Maurer-Cartan equation has two terms:
    \begin{itemize}
        \item The first term $m^1(a) = d_A a + a d_A$. The vanishing of this term states that $a$ is a chain map;
        \item The vanishing of the second term $m^2(a, a)$ tells us that $a$ squares to zero (so that it gives a differential).
    \end{itemize}
    The combination of these two terms checks the condition that $(d_A+a)\circ (d_A+a)=0$; that is that we can deform the differential by $a$.
 