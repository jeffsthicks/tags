%label:"art:tropicalCompactificationSketch"
%type:"article"
%name:"a sketch of compactification in tropical geometry"
%parent:"art_tropicalGeometrySeminarOutline"


Note that we have some ideas of compactification already: after all we can define tropical polynomials on $\RR^n$, and $\TT^n$ is a compactification of $\RR^n$.
The space $\TT^n$ has a decomposition into pieces $\RR_I$, which we call the torus orbits. The key notion that we will need to study is how many ``-'infinity'''s does a point $x$ belong to. 
%label:"def:sedentarity"
%type:"definition"
%name:"sedentarity"


    Given $x\in \TT^n$, we define the \emph{sedentarity} of $x$ to be the number of $-\infty$'s belonging to $q$ in its coordinate description. 


We say that $\sigma\in \TT^n$ is a polyhedron if it is the closure of a polyhedron in one of the torus orbits of $\TT^n$. 
From these compact pieces, we can define a tropical variety: a locally open subset of polyhedra satisfying a balancing condition. 