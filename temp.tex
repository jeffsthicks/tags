
\label{art:heegaardFloerExercises}

\begin{exercise}
    \label{exr:sym2s1}
     What is $\Sym^2(S^1)$?
\end{exercise}
\begin{exercise}
    \label{exr:homologyFromHeegaardDiagram}
    Let $(\Sigma, \{\alpha_i\}_{i=1}^g, \{\beta_i\}_{i=1}^g)$ be a Heegaard diagram for $M$.. Let $A$ be the $g\times g$ matrix whose entries over $\ZZ/2\ZZ$ given by  
    \[a_{ij}=\alpha_{i}\cap \beta_j \mod 2.\] 
    Given the matrix $A$, compute $H^\bullet(M, \ZZ/2\ZZ)$.
\end{exercise}
\begin{exercise}
    \label{exr:homologyClassOfWhitneyDisk}
    \begin{enumerate}
            \item    Let $\Sigma$ be a surface. Show that $H_1(\Sym^g(\Sigma))\cong H_1(\Sigma)$.
            \item    Show that $H_1(\Sym^g(\Sigma))/(H_1(T_\alpha)\oplus H_1(T_\beta)\cong H_1(\Sigma))/\langle\{\alpha_i,\beta_j\}_{i,j}\rangle$
             \item   What is the homology class of the boundary of a Whitney disk?
    \end{enumerate}
\end{exercise}    
\begin{exercise}
    \label{exr:heegaardFloerLensSpace}
        A \emph{Lens space} $L(p,q)$ can be defined as the $3$-manifold with a Heegaard splitting of genus 1, with an $\alpha$-curve given by a meridian of the the torus, $\beta$-curve given by the longitude of the torus. Use the above obstruction to compute $\widehat{\HHF}(L(p,q))$.
        \todo{put in diagram of lens space}
\end{exercise}
        
\begin{exercise}
\label{exr:heegaardMoves}
Compute the Heegaard Floer homology of the manifold with Heegaard splitting as shown, from the diagram. What is the manifold?
\input{fig_heegaardMysteryDiagram}
\end{exercise}
\begin{exercise}
     \begin{enumerate}
            \item Let $Y$ be a $3$-manifold with Heegaard splitting $(\Sigma,\mathbf{\alpha},\mathbf{\beta})$. Pick  orientations on $\Sigma,\alpha,\beta$. Explain how this induces a $\Z/2$ grading on $\widehat{\HHF}(Y)$.
            
            \item Let $\Q$ be a rational homology sphere. Show that 
            \[\chi(\widehat{\HHF}(Y)):=\rank(\widehat{\HHF}_0(Y))-\rank(\widehat{\HHF}_1(Y))=|H_1(Y;\Z)|.\]
            
            \item A rational homology sphere (i.e. a 3 manifold with $H_*(Y)\cong H_*(S^3)$) is called an \emph{L-space}, if $\rank(\widehat{\HHF}(Y))$ is minimal in the sense that $\rank(\widehat{\HHF}(Y))=|H_1(Y)|$. Explain why Lens spaces are $L$-spaces
        \end{enumerate}
\end{exercise}

\begin{exercise}
        \begin{itemize}
            \item Consider the ``decatigorification" of $\widehat{\HHF}$ given by the Euler characteristic as described in the previous question. Show that it is invariant under Heegaard moves. How might the proof of invariance of $\widehat{\HHF}$ differ?
        \item  Suppose $Y_1,Y_2$ are $3$-manifolds. How are $\widehat{\HHF}(Y_1\# Y_2),\widehat{\HHF}(Y_1),$ and $\widehat{\HHF}(Y_2)$ related?
    \item Show that $\widehat{\HHF}$ is invariant under stabilization and Hamiltonian isotopy in the sense that the resulting groups are isomorphic.
\end{exercise}
    