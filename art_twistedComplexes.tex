%label:"art:twistedComplexes"
%author:JeffHicks
%name:"the category of twisted complexes"
%type:"article"


One viewpoint on mapping cones of cochain  complexes is that they give \emph{deformations} of (direct sums of) objects of our categories. Given a map of cochain complexes $f: A\to B$, the differential on $\cone(f)$ has the form 
\[d_{\cone(f)}=\begin{pmatrix} d_A & 0 \\ 0 & d_B \end{pmatrix} + \begin{pmatrix} 0 & 0\\ f & 0\end{pmatrix}\]
where the first term is the differential on $A\oplus B[1]$, and the second term ``deforms'' the differential on this chain complex.

Twisted complexes  extend this story in several directions: firstly, we expand the set of deformations so that the objects we consider are chain complexes up to homotopy, and we allow deformations of the product (and not only differential) structure.

%tag:000L
%label:"def:twistedComplex"
%author:JeffHicks
%name:"twisted complex"
%type:"definition"

 
 
Let $\mathcal C$ be an $A_\infty$ category. A twisted complex $(E, \delta_E)$ consists of:
\begin{itemize}
\item $E$, a formal direct sum of shifts of objects
\[E:=\bigoplus_{i=1}^N E_i[k_i]\]
where $E_{i}\in Ob(\mathcal C)$, and $k_i\in \ZZ$
\item  A differential $\delta_E$, which can be written as a matrix  $\delta^{ij}_E: E_i[k]\to E_j[k-1]$ of degree 1 maps. These maps must satisfy the following conditions:
\begin{itemize}
    \item the matrix $\delta_E$ is strictly upper triangular and;
    \item They satisfy the Maurer-Cartan relation:
    \[\sum_{k\geq 1} m^k (\delta\otimes\cdots \otimes \delta) =0.\]
\end{itemize}
\end{itemize}

 

%label:"rem:twistedComplexConvergence"
%author:JeffHicks
%name:"convergence in twisted complexes"
%type:"remark"


The condition that the matrix $\delta_E$ is strictly upper triangular is to ensure that the sum in the Maurer-Cartan relation converges. One can also ask that there exists a filtration on $E$, the formal direct sum of shifts of objects, and that the differential $\delta_E$ respects the filtration (see Section 31 of \cite{seidel2008fukaya}). From this perspective, the twisted complex looks more like a formal deformation of the direct sum.
 
%label:"rem:twistedComplex"
%author:JeffHicks
%name:"properties of twisted complexes"
%type:"remark"




    With regards to the first point: Suppose that we have a (not necessarily exact) sequence of chain complexes $A\xrightarrow{f} B \xrightarrow{g} C$. The total complex of this sequence will not be a chain complex (as $g\circ f \neq 0$). However, to build a twisted complex from this data we will only need that $g\circ f$ is homotopic to zero. Suppose that  $H:A\to C[1]$ is a homotopy (so that $d_AH+Hd_C=g\circ f$). Then  
    \[\delta = \begin{pmatrix}
        0 & 0 & 0\\
        f & 0 & 0\\
        H & g & 0
    \end{pmatrix}\]
    gives us a twisted complex on $A\oplus B[1]\oplus C[2]$.

    For the second point: Let $(A, m^k)$ be an $A_\infty$ algebra. There are a particularly nice class of deformations of $A_\infty$ governed by elements $a\in A^1$ satisfying the Maurer-Cartan equation:
    \[m^1(a)+m^2(a\otimes a)+m^3(a\otimes a \otimes a)+\cdots =0.\]
    In order for this equation to make sense, one needs show that the sum converges. This is usually achieved by asking that $A$ be filtered and that $m^k(a^{\otimes k})$ lies increasingly positive filtration levels.
    When one can make sense of this equation, we can define a new $A_\infty$ algebra, $(A, m^k_a)$ whose product is defined by 
    \[m^k_a:=\sum_{n>0}\sum_{j_0+\cdots+j_k=n} m^{k+n}(a^{\otimes j_0}\otimes \id \otimes a^{\otimes j_1}\otimes \id \cdots \otimes a^{\otimes j_{k-1}}\otimes \id \otimes a^{\otimes j_k})\]

    Now consider the setting where $C$ is a chain complex, and $A=\hom(C, C)$. Then $a\in A^1$ corresponds to a map $a: C\to C[1]$, and the Maurer-Cartan equation has two terms:
    \begin{itemize}
        \item The first term $m^1(a) = d_A a + a d_A$. The vanishing of this term states that $a$ is a chain map;
        \item The vanishing of the second term $m^2(a, a)$ tells us that $a$ squares to zero (so that it gives a differential).
    \end{itemize}
    The combination of these two terms checks the condition that $(d_A+a)\circ (d_A -a)=0$; that is that we can deform the differential by $(-1)^k a$.
 
%label:"def:morphismOfTwistedComplexes"
%author:JeffHicks
%name:"morphism of twisted complexes"
%type:"definition"

 
 
Let $(E, \delta_E)$ and $(F, \delta_F)$ be two twisted complexes. A \emph{morphism of twisted complexes} is a collection of morphisms  of $\mathcal C$
\[f_{ij}:E_i[k_i]\to E_j[k_j].\]
The set of morphisms can therefore be written as $\hom^d((E, \delta_E), (F, \delta_F))=\bigoplus_{i, j}\hom^{d+k^F_j-k^E_i}(E_i, F_j).$
Given a sequence $\{(E_i, \delta_i)\}_{i=0}^k$ of twisted complexes, and $a_i\in\hom^d((E_{i-1}, \delta_{E_{i-1}}), (E_{i}, \delta_{E_{i}}))$, we have a composition 
\[m^k_{\operatorname{Tw}}(a_{k}\otimes \cdots \otimes a_{1} ):=\sum_{j_0, \ldots, j_k\geq 0} m^k(\delta_k^{\otimes j_k}\otimes a_{k}\otimes \delta_{k-1}^{\otimes j_{k-1}}\otimes a_{k-1}\otimes \cdots \otimes a_1\otimes \delta_0^{\otimes j_{0}}).\]

 
%tag:000T
%label:"prp:categoryOfTwistedComplexes"
%author:JeffHicks
%name:"category of twisted complexes"
%type:"proposition"

 
 
Let $\mathcal C$ be an $A_\infty$ category. The category of twisted complexes, $\operatorname{Tw}(\mathcal C)$, is the $A_\infty$ category whose objects are twisted complexes, morphisms are morphisms of twisted complexes, and $A_\infty$ compositions are given by $m^k_{\operatorname{Tw}}$.

 
%tag:000U
%label:"thm:categoryOfTwistedComplexesIsTriangulated"
%author:JeffHicks
%name:"twisted complexes form a triangulated category. "
%type:"theorem"


Let $\mathcal C$ be an $A_\infty$ category. The category of twisted complexes, $\operatorname{Tw}(\mathcal C)$ is a triangulated category. There is a fully faithful inclusion $\mathcal A\to \operatorname{Tw}(\mathcal C)$. Furthermore, the image of $\mathcal A$ generated $\operatorname{Tw}(\mathcal C)$.

%label:"prf:categoryOfTwistedComplexesIsTriangulated"
%author:JeffHicks
%name:"twisted complexes form a triangulated category. "
%type:"proof"
%parent:thm:categoryOfTwistedComplexesIsTriangulated
%completeness:5

 
 To give twisted complexes the structure of a triangulated category, we must specify what the exact triangles are. Given a morphism $f: (E, \delta_E)\to (F, \delta_F)$, we can define the cone of $f$ to be the twisted complex $(E[1]\oplus F, \delta')$ where $\delta'$ is the matrix
 \[
    \left(\begin{array}{c|c} \delta_E &0 \\ \hline f^\delta_F  & \delta_F\end{array}\right).
\]

 
There exists an inclusion functor $i:\mathcal C\to \Tw(\mathcal C)$.
We can therefore declare that the triangle $A\to B\to C$ is exact in  $\mathcal A$ is if $C$ is quasi-isomorphic to $\cone(A\to B)$ in the category of twisted complexes. 

