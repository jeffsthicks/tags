%label:"exm:LagrangiansGivenBySymplecticParallelTransport"
%type:"example"
%name:"Lagrangians given by symplectic parallel transport"
%caption:""
%parent:"art_lefschetzExpanded"


    Recall our running example $\pi:\CC^2\to \CC$ from \cref{exm:LocalModelOfASingularity}. 
    We will prove that the symplectic parallel transport map preserves a class of Lagrangian submanifolds of the fiber.
    Consider the function $H(z_1, z_2)= \frac{1}{2}\left(|z_1|^2-|z_2|^2\right)=\frac{1}{2}\left( x_1^2+y_1^2-x_2^2-y_2^2\right)$. 
    The exterior derivative of this function, in local coordinates, is given by 
    \[dH= x_1dx_1 +y_1dy_1 -x_2dx_2- y_2dy_2.\]
    We prove that $H$ is invariant under the action of symplectic parallel transport along the fibration $\pi:\CC^2\to \CC$. 
    In this example, we can explicitly compute that $H$ is invariant under vectors contained in $\ker(d\pi)^{\omega_\bot}$. 
    
    The kernel of $d\pi=z_2dz_1+z_1dz_2$ at a point $(z_1, z_2)$ is the complex subspace generated by the vector \begin{align*}
        \ker_{(z_1, z_2)}(d\pi)=&\Span_\CC(\langle z_1, -z_2\rangle)\\
        =&\Span_\RR(\langle x_1, y_1, -x_2, -y_2\rangle, \langle -y_1, x_1, y_2, -x_2\rangle ).
    \end{align*}
    In this setting, the symplectic complement is described by the orthogonal complement, and so 
    \begin{align*}
        (\ker_{(z_1, z_2)}(d\pi))^{\omega\bot}=&\Span_\CC(\langle \bar z_2, \bar z_1\rangle)\\
        =&\Span_\RR(\langle x_2, -y_2, x_1, -y_1\rangle, \langle y_2, x_2, y_1, x_1\rangle ).
    \end{align*}
    One then checks that $dH$ vanishes on this by computing $dH(v)=0$ for $v\in (\ker_{(z_1, z_2)}(d\pi))^{\omega\bot}$
    \begin{align*}
        \langle x_1, y_1, -x_2,- y_2\rangle\cdot \langle x_2, -y_2, x_1, -y_1\rangle=&0\\
        \langle x_1, y_1, -x_2,-y_2\rangle\cdot \langle y_2, x_2, y_1, x_1 \rangle=&0
    \end{align*}

    This means that the level sets of $H$ are preserved under parallel transport.
    We use to this to describe some Lagrangian submanifolds of $\CC^2$. 

    If we take a level set of $H$ and restrict to a fiber above the point $re^{\jmath c}$, the level set $H^{-1}(\lambda)\cap \pi^{-1}(re^{\jmath c})$ can be explicitly parameterized by $S^1:=\theta\mapsto re^{\jmath c}\cdot(s e^{\jmath\theta}, s^{-1} e^{-\jmath\theta})$, where $s$ is determined by $r^2(s^2-s^{-2})=\lambda$. 
    Simply because every curve is a Lagrangian submanifold of a $\CC^*$, the level set of $H$ restricted to a fiber of $\pi$ is a Lagrangian submanifold.
    
    We can now apply \cref{prp:ParallelTransportOfLagrangianSubmanifold} to obtain some new Lagrangian submanifolds of $\CC^2$ from parallel transport of these level sets. 
    Let $\gamma:[0, 1]\to \CC\setminus 0$ be a closed curve, and $\lambda\in \RR$ some value. 
    Define the Lagrangian $L_{\gamma, \lambda}$ to be the parallel transport of the $\lambda$-level set along the curve $\gamma$. 
    This already gives several interesting examples of Lagrangian submanifolds inside of $\CC^2$. 
    These Lagrangian submanifolds can also be characterized in the following way:
    \[L_{\gamma, \lambda}:=\{(z_1, z_2)\;|\; H(z_1, z_2)=\lambda, \pi(z_1, z_2)\in \Im(\gamma)\}.\]

    A good example of one of these Lagrangians is the \emph{product torus}. Let $\gamma_r=re^{\jmath\theta}$.
    Let $s$ be the real value so that $r^2(s^2-s^{-2})=\lambda$.  Then the Lagrangian $L_{\gamma_r, \lambda}$ is explicitly parameterized by:
    \[L_{\gamma_r, \lambda}=\{r(se^{\jmath\theta_1}, se^{\jmath\theta_2})\}\]
    This agrees with the definition of the product torus from \cref{exm:productTorus}.

