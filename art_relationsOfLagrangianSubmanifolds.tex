%label:"art:relationsOfLagrangianSubmanifolds"
%type:"article"
%name:"relations between Lagrangian submanifolds"

In these notes, we will attempt to shed a little light on the construction and classification of Lagrangian submanifolds. 
Even in the simplest examples, this classification can become quite difficult. 
To give a meaningful answer to this question, we'll need a notion of equivalence between Lagrangian submanifolds. 
For submanifolds, a natural equivalence to consider is the isotopy class or the homotopy class.

Let $\li_t: L\times I\to X$ be a homotopy of Lagrangian submanifolds.
We now describe the \emph{flux class} of the homotopy, which is an element $\Flux_{\li_t}\in H^1(L;\RR)$.
To define this class we prescribe its values on chains of $L$. 
Let $c\in C_1(L)$ be a chain. 
Then the homotopy can be applied to $c$ to give a 2-chain $\li_t(c)\in C_2(X)$. 
%label:"def:flux"
%type:"definition"
%name:"flux of Lagrangian homotopy"


    The \emph{flux} of a Lagrangian homotopy $\li_t:L\times I\to X$ is the cohomology class defined by
    \[\Flux_{\li_t}(c):=\int_{\li_t(c)}\omega.\]


To show that this is a cohomology class, we need to show that the flux homomorphism $\Flux_{\li_t}$ vanishes on boundaries $\partial b\in C_1(L_0)$ 

Without loss of generality, suppose that $I=[0,1]$, and let $b\in C_2(L)$ be a 2-chain.
Then $\li_t(b)$ similarly gives a 3-chain in $X$ whose boundary components satisfy the relation $\li_t(\partial b)=i_0(b)-i_1(b)+\partial(\li_t(b))$. 
Applying the definition of the flux homomorphism to this relation yields:
\[
    \Flux_{\li_t}(\partial b)=\int_{\li_t(\partial b)}\omega=\int_{i_0(b)}\omega-\int_{i_1(b)}\omega + \int_{\partial(\li_t(b))}\omega\]
The first two terms vanish because $i_0(b)$ and $i_1(b)$ are subsets of a Lagrangian submanifold so $\omega|_{\li_0(L)}=\omega|_{\li_1(L)}=0$.
Because $\omega$ is closed, an application of Stoke's theorem shows that the last term vanishes as well. 
%label:"exm:fluxInCylinder"
%type:"example"
%name:"flux on $\CC^*$"


    Let $\CC^*$ be equipped with the symplectic form $\omega=\frac{1}{2\pi} d(\log r) \wedge d\theta$. 
    Let $L_r$ be the Lagrangian  
        \[L_r:=\{z \text{ such that } \log|z|=r\}.\]
    Let $\li_r:L\times [r_0, r_1]\to X$ be the isotopy between $L_{r_0}$ and $L_{r_1}$. 
    Let $e\in H_1(L_r)$ be the fundamental class of $L$. 
    The amount of flux swept out between these two Lagrangian submanifolds is given by the difference of their $r$-values. 
    \[\Flux_{i_r}(e)=\int_{S^1} \int_{r_0}^{r^1} \frac{1}{2\pi} d(\log |z|)\wedge d\theta = r_1-r_0.\]
    For this reason, the value $r$ is sometimes called the ``flux coordinate'' of the fibration $\CC^*\to \RR$.
    The flux class therefore provides a nice parameterization of the space of Lagrangian submanifolds in this example. 


%label:"fig:fluxInCylinder"
%author:JeffHicks
%name:"flux for $\CC^*$"
%type:"figure"
%parent:"exm:fluxInCylinder"
%caption:"A basic computation of flux swept out between two submanifolds of $\CC^*$"

\begin{tikzpicture}

    \fill[fill=gray!20, path fading = west]  (-4.5,1.5) rectangle (-2.5,-0.5);
    \fill[fill=gray!20]  (-2.5,-0.5) rectangle (2.5,1.5);
    
    \fill[fill=gray!20, path fading = east]  (4.5,1.5) rectangle (2.5,-0.5);
    \fill[fill=blue!20]  (-2.5,1.5) rectangle (2.5,-0.5);
    
    \begin{scope}[shift={(2.5,0)}, draw=gray]
    
    \fill[fill=blue!20]  (0,0.5) ellipse (0.5 and 1);
    \begin{scope}[]
    
    \clip  (0,2) rectangle (1,-1);
    \draw[red, fill=blue!20]  (0,0.5) ellipse (0.5 and 1);
    \end{scope}
    \begin{scope}[]
    
    \clip  (0,2) rectangle (-1,-1);
    \draw[dashed, red]  (0,0.5) ellipse (0.5 and 1);
    \end{scope}
    \end{scope}
    
    \begin{scope}[shift={(-2.5,0)},draw=gray]
    
    \begin{scope}[]
    
    \clip  (0,2) rectangle (1,-1);
    \draw[red]  (0,0.5) ellipse (0.5 and 1);
    \end{scope}
    \begin{scope}[]
    \fill[ fill=blue!10]  (0,0.5) ellipse (0.5 and 1);
    
    \clip  (0,2) rectangle (-1,-1);
    \draw[dashed, red]  (0,0.5) ellipse (0.5 and 1);
    \end{scope}
    \end{scope}
    \begin{scope}[]
    \draw (-3.5,1.5) -- (3.5,1.5);
    \begin{scope}[]
    \draw[dashed] (-3.5,1.5) -- (-5,1.5);
    \end{scope}
    \begin{scope}[shift={(8.5,0)}]
    \draw[dashed] (-3.5,1.5) -- (-5,1.5);
    \end{scope}
    \end{scope}
    \begin{scope}[shift={(0,-2)}]
    \draw (-3.5,1.5) -- (3.5,1.5);
    \begin{scope}[]
    \draw[dashed] (-3.5,1.5) -- (-5,1.5);
    \end{scope}
    \begin{scope}[shift={(8.5,0)}]
    \draw[dashed] (-3.5,1.5) -- (-5,1.5);
    \end{scope}
    \end{scope}
    \node at (-2.5,2) {$\log|z|=r_1$};
    \node at (2.5,2) {$\log|z|=r_2$};
    \node at (0,0.5) {Flux Swept};
    \node at (-2.5,2.5) {$L_0$};
    \node at (2.5,2.5) {$L_1$};
    \node at (4,0.5) {$\mathbb C^*$};
\end{tikzpicture}
%label:"exm:fluxInCotangent"
%type:"example"
%name:"flux in $T^*L$"


    Consider the symplectic manifold $X=T^*L$.
    Consider a closed one form  $\eta \in \Omega^1(L, \RR)$.
    Use this to create a Lagrangian isotopy 
    \begin{align*}
        \li_t: L\times [0, 1]\to& T^*L &&       (q,t) \mapsto& t\cdot \eta_p
    \end{align*}
    between the zero section and the graph of $\eta$. 
    To each loop $\gamma\in L$, look at the cylinder $\li_t\circ \gamma:S^1\times I\to X$.
    This can be explicitly parameterized by 
    \begin{align*}
        c: S^1\times I \to& T^*L&&
        (\theta, t) \mapsto& (\gamma(\theta), t\cdot \eta_{\gamma(\theta)}).
    \end{align*}
    The cotangent bundle has an exact symplectic form, $\omega=d\lambda$, so computing the flux can be simplified by using Stoke's theorem:
    \begin{align*}
        \Flux_{\li_t}(\gamma)=&\int_{c}\omega=\int_{c}d\lambda \\
        =&\int_{S^1\times \{1\}}c^*\lambda - \int_{S^1\times\{0\}}c^*\lambda \\
        =&\int_{S^1}\gamma^*\eta=\eta(\gamma)
    \end{align*}
    So $[\Flux_{\li_t}]=[\eta]\in \Omega(L)$. 


This example highlights two important properties of Lagrangian submanifolds, and Lagrangian homotopy. 
First, there homotopies a stronger notion of equivalence for Lagrangian submanifolds beyond homotopy by looking at those homotopies which sweep out zero flux. 
%label:"def:exactHomotopy"
%type:"definition"
%name:"exact homotopy of Lagrangian submanifolds"


    We call $i_t: L\times I \to X$ an \emph{exact homotopy} if for every subinterval $J\subset I$, the flux of the restriction is exact,
    \[[\Flux_{i_{t}}|_J]=0\in H^1(L).\] 


\snip{Similar to the case of Hamiltonian isotopies}{prp:exactSymplecticHamiltonian}, exact isotopies of Lagrangian submanifolds can be characterized in terms of the Lagrangian isotopy itself. 
%label:"clm:hamiltonianFlux"
%type:"claim"
%name:"Hamiltonian associated to exact homotopy"


    Let $\li_t: L\times I\to X$ be a Lagrangian homotopy. 
    This is a exact homotopy if there exists a Hamiltonian function $H_t:L\to \RR$ such that for all $v\in TL$, 
    \[\omega\left(\frac{d}{dt}\li_t,v\right)=dH_t(v).\]


%label:"prf:hamiltonianFlux"
%type:"proof"
%name:"Hamiltonian flux"


    We show one direction here, which is that  $\omega\left(\frac{d}{dt}\li_t,v\right)=dH_t(v)$ implies  $\Flux_{\li_t}$ vanishes in cohomology. 
    Let $c:S^1\to L$ be any 1-cycle in $L$.
    Parameterize a 2-chain $\li_t\circ c: S^1\times I \to X$ with coordinates $(\theta, t)$. 
    The flux class applied to $c$ can be explicitly computed: 
    \begin{align*}
        \Flux_{\li_t}(c)=&\int_{\li_t\circ c}\omega
         =\int_{I \times S^1} (\li_t\circ c )^* \omega\\
        =&\int_I \int_{S^1} c^*\circ (\li_t)^* \omega
         =\int_I \int_{S^1} (c^*  \iota_{\frac{d}{dt}\li_t}\omega ) dt\\
        =&\int_I \left(\int_{S^1} (c^* dH_t) \right)dt
         =\int_I \left(\int_{S^1} d(c^*H_t)\right) dt
    \end{align*}
    By Stoke's theorem, the integral of an exact form over the circle is zero. 
   
    For the reverse direction, fix a base point $x_0\in L$. 
    For every point $x\in L$, pick a path $\gamma_x: [0,1]\to L$ with $\gamma_x(0)=x_0$ and $\gamma_x(1)=x$.
    Define the function $H_t: L\to \RR$ by 
    \[dH_t(x_1):=\int_{\li_t\circ \gamma} \omega.\]
    Because the flux of the isotopy is zero, this integral does not depend on the choices of paths $\gamma_x$ and gives a well defined function on $L$.
    We now show that this function generates the Lagrangian isotopy.
    The vector field $\frac{d}{dt}\li_t$ is determined by the form $\iota_{\frac{d}{dt}\li_t}\omega$. 
    Since $\iota_{\frac{d}{dt}\li_t}$ is closed,  
    \begin{align*}
        \int_{\gamma_x} \iota_{\frac{d}{dt}\li_t}\omega =& 
    \end{align*}
    \todo{We now check that this these two things match up by  computing the vector field.}


This shows that when $L$ is embedded, a time dependent Hamiltonian $H_t: L\times (-\epsilon, \epsilon)\to \RR$ defines an isotopy of embeddings $\li_t: L\times(-\epsilon, \epsilon)\to X$ for small values of $t$.
We say this is the isotopy generated by the Hamiltonian $H$.
Provided that $L$ is compact, we immediately get the following statement on the non-displacibility of Lagrangian submanifolds by small exact Lagrangian homotopies. 
%label:"prp:lagrangianIntersection"
%type:"proposition"
%name:"intersections for small Hamiltonians"


    Let $H: L\to \RR$ be a non-time dependent Hamiltonian .
    Let $\li_t: L\times[0, c]\to X$ be the isotopy generated by $H$. 
    There exists $\epsilon>0$ such that for all $t_0, t_1\in [0, \epsilon]$, 
    \begin{align*}
        |\li_{t_0}(L)\cap \li_{t_1}(L)|\geq \sum_{i=1}^n b_i(L).
    \end{align*}


%label:"prf:lagrangianIntersection"
%type:"proof"
%name:"intersections for small Hamiltonians"


    Let $\Crit(H)$ be the set of critical points of $H$.
    From \cref{clm:hamiltonianFlux}, whenever $x\in \Crit(H)$, $\frac{d\li}{dt}(x)=0$ and so $\li_{t_0}(x)=\li_{t_1}(x)$.
    Therefore, $|\li_{t_0}(L)\cap \li_{t_1}(L)|\subset\Crit(H)$.
    The proposition follows from the Morse inequalities. 


In fact, this can extends to the case of time-dependent Hamiltonians for sufficiently small $t$. 
\begin{corollary}
    Let $H_t: L\to \RR$ generate a Hamiltonian isotopy. 
    Then there exists $\epsilon>0$ such that for all $t_0, t_1\in [0,\epsilon)$, 
    \[|\li_{t_0}(L)\cap \li_{t_1}(L)|\geq \sum_{i=1}^n b_i(L).\]
\end{corollary}
Secondly, every cohomological class is realizable as a flux class in the cotangent bundle. 
This observation motivates the following theorem, which shows that this behavior occurs more generally.
%label:"thm:weinsteinNeighborhood"
%author:JeffHicks
%name:"Weinstein Neighborhood Theorem"
%type:"theorem"
%source:"weinstein1971symplectic"


Let $\li: L\to X$ be a compact Lagrangian submanifold of $(X,  \omega)$. Then there exists a neighborhood of the zero section $T^*_\epsilon L\subset TL$ and a symplectic embedding $\phi: T^*_\epsilon L\to X$ which agrees on the zero section, in the sense that 
\[\phi|_L=\li.\]
In other words,  a symplectic neighborhood of a Lagrangian submanifold is determined by the diffeomorphism class of the Lagrangian. 
This local model will be useful later, as many of our constructions involving Lagrangian submanifolds will involve restricting to a Weinstein neighborhood, performing the construction there, and then implanting our construction into the original symplectic manifold.
An example of such a construction is the following.
%label:"lem:straightening"
%type:"lemma"
%name:"standard model for Lagrangian intersection"


    Let $L\subset T^*Q$ be a Lagrangian submanifold.
    Suppose that the intersections between $L$ and $Q$ are transverse.
    Then there exists a Hamiltonian isotopy $\phi$ of $L$ so that
    \[\phi(L)\cap B^*_\epsilon Q=\bigcup_{p\in L\cap Q} T^*_pQ.\]


%label:"prf:straightening"
%type:"proof"
%name:"local model for Lagrangian intersection"


    At each $q\in L\cap Q$, consider the Lagrangian submanifold $T^*qQ$. 
    Take local coordinates $(q_1, \ldots, q_n, p_1, \ldots, p_n)$ identifying $q$ with the origin so that $T^*qQ$ is the linear subspace in the $p_i$ directions.
    We can take a Weinstein neighborhood $T^*_\epsilon(T^*_qQ)$ of $T^*_qQ$, whose cotangent bundle structure is 
    \begin{align*}
        T^*_\epsilon(T^*_qQ)\to T^*qQ && (q_1, \ldots, q_n, p_1, \ldots, p_n)\mapsto (p_1, \ldots, p_n).
    \end{align*}
    Since the intersection $L\cap Q$ is transverse, the projection $T_qL\to T_q(T^*qQ)$ is surjective. Therefore when restricted to a small enough neighborhood of $\in U\subset T^*_qQ$  the Lagrangian $L|_{T^*_\epsilon U}$ presents itself as a section of $T^*_\epsilon U\to U$. 
    Therefore, there exists a one form $\eta\in \Omega^1(U)$ so that $L|_{T^*U}$ is parameterized by $(p, \eta_p)$.

    By taking an even smaller $U$, we may assume that $U$ is a contractible neighborhood, and $\eta=dH$ is an exact one-form.
    Pick $\rho$ a function which vanishes in a neighborhood of $q\in U$, and takes the value $1$ in a neighborhood of $\partial U$. 
    Consider the Lagrangian section of $T^*U$ parameterized by $d(\rho \cdot H)$.
    This section is Hamiltonian isotopic to $L|_{T^*_\epsilon U}$ relative boundary.
    Additionally, $d(\rho\cdot H)$ agrees with $U=T^*_qQ$ in a small neighborhood of $q$.
    The Lagrangian submanifold $L\setminus (L|_{T^*U})\cup (d(\rho\cdot H))$ is Hamiltonian isotopic to $L$. 


